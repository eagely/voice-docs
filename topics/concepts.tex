\cleardoubleemptypage
\renewcommand*\chapterpagestyle{scrheadings}

\chapter{Concepts and Theory}
It is important to understand the main concepts behind the voice assistant,
to be able to properly comprehend the explanations that follow further:

\section{Voice Assistant}
A voice assistant is a software application that enables users to receive assistance and interact with their system using voice commands.
using voice commands. More technically, it is a voice-controlled system that utilizes speech recognition and
natural language processing to process user input and respond to it,
be that by playing audio, displaying text, or executing actions.

\section{Artificial Intelligence}
Artificial Intelligence\footnote{Artificial Intelligence \cite{ai}} is a broad subset of computer science
with its main focus being on creating systems capable of performing tasks,
which would usually require the human mind --- or natural intelligence.
Such tasks include things like reasoning, making decisions, learning from big datasets and understanding language.
Over the years, many approaches to creating artificially intelligent systems have been created,
from the simplest rule-based parsers and logical reasoning to more modern techniques
featuring learning processes on giant datasets scraped from the internet.

\section{Machine Learning}
Machine Learning\footnote{Machine Learning \cite{ml}} is a subset of artificial intelligence  
that aims to develop algorithms and methods which allow computers to increase
their reasoning performance over time without having to reprogram them constantly.
There are a lot of machine learning techniques, some more effective than others,
but it usually boils down to learning from big datasets and finding
patterns in the data which allow the model to make predictions based on those patterns.

\section{Natural Language Processing}
Natural Language Processing\footnote{Natural Language Processing \cite{nlp}} is a subfield of artificial intelligence  
and is closely related to the field of computational linguistics\footnote{Computational Linguistics \cite{cl}}.  
The goal is to automatically understand human language and translate it into  
a structured form, which can be achieved using various approaches,  
ranging from simple pattern matching\footnote{Formal Grammar \cite{pm}}  
to more advanced systems that utilize machine learning.

\section{Large Language Model}
A large language model\footnote{Large Language Model \cite{llm}}
is a more advanced type of machine learning system;
it is designed to "understand" natural language at scale,
so it is a sort of natural language processing system.
They are trained on vast amounts of text, usually obtained by scraping the internet,
which enables them to produce coherent, human-like conversation.
A large language model is an advanced type of machine learning system  
designed to understand and generate natural sounding language at scale.
