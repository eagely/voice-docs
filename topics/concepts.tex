\cleardoubleemptypage
\renewcommand*\chapterpagestyle{scrheadings}

\chapter{Concepts and Theory}
The concepts and theory behind the voice assistant are discussed in this chapter.

\section{Voice Assistant}
A voice assistant is a software application that enables users  
to interact with a device using voice commands.  
It is a form of \textit{voice-controlled system} that utilizes speech recognition  
and natural language processing techniques to interpret and respond  
to user input,  
be that by playing audio, displaying text, or executing actions.

\section{Artificial Intelligence}
Artificial Intelligence\footnote{Artificial Intelligence \cite{ai}} is a wide-ranging branch  
of computer science that focuses on building systems capable of performing  
tasks that usually require human thought.  
These tasks include reasoning, decision-making, learning, and understanding  
language. Approaches in artificial intelligence range from traditional rule-based methods  
and logical reasoning to modern techniques involving computer-based learning processes.

\section{Machine Learning}
Machine Learning\footnote{Machine Learning \cite{ml}} is a subset of artificial intelligence  
that develops algorithms allowing computers to learn from data and improve  
their performance over time without explicit reprogramming.  
Techniques in machine learning include methods that learn from examples,  
explore data patterns, and make predictions based on that information.

\section{Natural Language Processing}
Natural Language Processing\footnote{Natural Language Processing \cite{nlp}} is a subfield of artificial intelligence  
and is closely related to computational linguistics\footnote{Computational Linguistics \cite{cl}}.  
The goal is to automatically understand human language and translate it into  
a structured form, which can be achieved using various approaches,  
ranging from simple pattern matching\footnote{Formal Grammar \cite{pm}}  
to more advanced systems that utilize machine learning.

\section{Large Language Model}
A large language model\footnote{Large Language Model \cite{llm}}
is a more advanced type of machine learning system;
it is designed to "understand" natural language at scale,
so it is a sort of natural language processing system.
They are trained on vast amounts of text, usually obtained by scraping the internet,
which enables them to produce coherent, human-like conversation.
A large language model is an advanced type of machine learning system  
designed to understand and generate natural sounding language at scale.
