\renewcommand*\chapterpagestyle{scrheadings}
\chapter{Costs}

\section{Calculations}

\begin{tabular}{|c|c|}
  \hline
  \textbf{Product} & \textbf{Costs} \\ \hline
  PCB v1.0 & 19,95€ \\ \hline
  Case v1.0 & 1€ \\ \hline
  Raspberry Pi 4 & 68,10€ \\ \hline
  IPS Display & 66,54€ \\ \hline
  AGPTEK Microphone & 13,10€ \\ \hline
  Case v2.0 & 0,50€ \\ \hline
  LM2576 & 10,21€ \\ \hline
  BMS 2S Li-Ion Battery Pack & 6,04€ \\ \hline
  18650 battery holder case & 10,07€ \\ \hline
  DC socket & 10,07€ \\ \hline
  PCB v2.0 & 28,95€ \\ \hline
  \textbf{Total} & \textbf{235,03€} \\ \hline

\end{tabular}

\vspace{0,5cm}
These are the expenses incurred for the diploma thesis. All costs were related exclusively to the hardware components. The materials were purchased and financed privately.


\section{Profitability analysis}

\begin{tabular}{|c|c|}
  \hline
  \textbf{Product} & \textbf{Costs} \\ \hline
  Case v1.0 & 1€ \\ \hline
  Raspberry Pi 4 & 30€ \\ \hline
  IPS Display & 30€ \\ \hline
  Microphone & 1€ \\ \hline
  Case v2.0 & 1€ \\ \hline
  LM2576 & 2€ \\ \hline
  BMS 2S Li-Ion Battery Pack & 2€ \\ \hline
  DC socket & 2€ \\ \hline
  PCB v2.0 & 1€ \\ \hline
  \textbf{Total} & \textbf{70€} \\ \hline

\end{tabular}

\newpage

For mass production, PCB v1.0 costs can be disregarded, as only PCB v2.0 will be utilized. The price of PCB v2.0 is €4 per batch of 5 pieces, with additional shipping costs. When ordering 100 pieces, shipping costs are estimated at €5 per batch of 5 units.

To analyse the profitability of the finished product, the cost per unit was calculated, taking into account both the development and production costs.

To calculate the development costs, the total working hours excluding time spent writing the documentation were multiplied by the average hourly wage:

\vspace{0.5cm}
\begin{center}
265\,h $\times$ 70\,€{}/h = 18550 \,€{}
\end{center}
\vspace{0.5cm}
 
The total hardware cost of one production unit equals 70\,€.  
The estimated time to manufacture 100 units is about 10 hours, hence an effective time spent per unit of 6 minutes.
The manufacturing cost is calculated based on the work required to assemble one unit and the typical man-hour rate for a minimum wage employee hired to assemble the product:

\vspace{0.5cm}
\begin{center}
10\,h $\times$ 10\,€{}/h = 100\,€{}
\end{center}

\begin{center}
100€ / 100 = 1\,€{}
\end{center}
\vspace{0.5cm}

The total cost of one production unit amounts to 71\,€. 
A reasonable price is 139.99\,€, 
which grants a profit margin of 97.2\%.

With this profit, the Break-Even-Point is calculated as follows:

\vspace{0.5cm}
\begin{center}
\[
\frac{18550\,€}{68.99\,€} \approx 269
\]
\end{center}
\vspace{0.5cm}

Thus, 269 pieces must be sold to break even with the development costs.

To reach a 130\% Return on Investment, the calculation is as follows:

\vspace{0.5cm}
\begin{center}
\[
\frac{18550\,€{} \times 2.3}{68.99\,€{}} \approx 619\,units
\]
\end{center}
\vspace{0.5cm}

Therefore, approximately 300 units must be sold to reach a 130\% ROI.