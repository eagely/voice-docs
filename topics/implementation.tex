\renewcommand*\chapterpagestyle{scrheadings}
\chapter{Implementation}

\section{Spring Boot Prototype}
The initial implementation of the backend service was developed in Kotlin using the Spring Boot framework\footnote{Spring Framework \cite{spring}}, a Java-based platform designed to provide foundational infrastructure for web applications. The service featured a prototype REST endpoint accessible at \texttt{http://localhost:8080/process}, which accepted textual input as an interim solution prior to the integration of audio processing capabilities. This endpoint utilized pattern-matching logic to direct weather-related queries to a dedicated weather service, while all other queries were processed using the GPT-4 language model.

\section{Rust Microservices Architecture}
As outlined in Section \ref{sec:design}, the system employs a modular microservices architecture where components can be swapped or updated independently.
This is achieved through Rust's trait system, which defines shared functionality for different structs, similar to interfaces in object-oriented languages.
Each microservice (e.g., audio transcription, weather data processing) implements a specific trait, ensuring consistent behavior across implementations.
The \texttt{async\_trait} crate enables asynchronous method support, critical for performance in I/O-bound tasks like audio processing or network requests.

\subsection{Recording}
The recording microservice trait includes two methods: \texttt{start} and \texttt{stop}.
The \texttt{start} method returns a result indicating success or failure, along with an error message if applicable.
The \texttt{stop} method returns the recorded audio bytes.

\begin{minted}{rust}
pub trait RecordingService {
    fn start(&self) -> Result<()>;
    fn stop(&self) -> Result<Bytes>;
}
\end{minted}

\subsection{Transcription}
This step involves transcribing audio bytes into text. Multiple options are available for this task,
including the open-source OpenAI Whisper for local processing or using Cloudflare Workers with Whisper.

\begin{minted}{rust}
#[async_trait]
pub trait TranscriptionService: Send + Sync {
    async fn transcribe(&self, audio: &Bytes) -> Result<String>;
}
\end{minted}

\subsection{Parsing}
Once the audio is transcribed into text, it is passed to a parser.
The parser can be a simple pattern matcher or an advanced LLM service such as Rasa.

\begin{minted}{rust}
#[async_trait]
pub trait ParsingService: Send + Sync {
    async fn parse(&self, input: &str) -> Result<Action>;
}
\end{minted}

This service accepts a string input, which it receives after the transcription process, and returns a structured \texttt{Action}.

\begin{minted}{rust}
pub struct Action {
    pub intent: Intent,
    pub entities: Vec<Entity>,
    pub text: String,
}
\end{minted}

An \texttt{Action} is a combination of intent and entities.
The intent represents the action that the parser determines should be performed, along with a confidence level.
This is particularly important for natural language processors using LLMs, where results are probabilistic rather than deterministic.

\begin{minted}{rust}
pub struct Intent {
    pub name: IntentKind,
    pub confidence: Option<f32>,
}

pub enum IntentKind {
    LlmQuery,
    WeatherQuery,
}
\end{minted}

Entities function similarly to variables. In a pattern-matching parser, they are straightforward variables,
while an LLM parser extracts what it deems the most likely desired variable,
such as the location in a weather request or the brightness level for adjusting lights.
When using an LLM parser, entities are also paired with a confidence level.
This confidence is typically very high (around 99.99\%), but it is wrapped in an \texttt{Option}
to account for scenarios where confidence is not applicable,
such as with a pattern-matching parser or a geopolitical entity extractor in an LLM processor.

\begin{minted}{rust}
pub struct Entity {
    pub entity: String,
    pub value: String,
    pub confidence: Option<f32>,
}
\end{minted}

\subsection{Geocoding}
The geocoding service converts textual location names into geographical coordinates:

\begin{minted}{rust}
pub trait GeocodingService: Send + Sync {
    async fn request(&self, address: &str) -> Result<GeocodeResponse>;
}
\end{minted}

The service returns standardized location data including the formatted name and coordinates:

\begin{minted}{rust}
pub struct GeocodeResponse {
    pub name: String,
    pub lat: String,
    pub lon: String,
}
\end{minted}

An implementation of the \texttt{GeocodingService} utilizes the OpenStreetMap Nominatim API
to convert unstructured location names into standardized geographical coordinates.
These coordinates serve as input parameters for subsequent \texttt{WeatherService} requests.
The API processes natural language location descriptions and returns precise latitude and longitude values along with a formatted location name.
This formatted name is later used as a display name when presenting weather response data to the client.

\subsection{Weather}
The weather service retrieves weather data using geographical coordinates:

\begin{minted}{rust}
pub trait WeatherService: Send + Sync {
    async fn request(&self, geocode: GeocodeResponse) -> Result<String>;
}
\end{minted}

The response data is structured according to the OpenWeather API\footnote{OpenWeather API \cite{openweathermap}} specification,
containing temperature, humidity, and weather condition descriptions:

\begin{minted}{rust}
pub struct WeatherResponse {
    pub main: Main,
    pub weather: Vec<Weather>,
}

pub struct Main {
    pub temp: f64,
    pub humidity: u8,
}

pub struct Weather {
    pub description: String,
}
\end{minted}

The OpenWeather API serves as the system's primary weather data provider, offering a free tier with 1,000 API calls per day.
It processes the geographic coordinates received from the \texttt{GeocodeResponse} generated by the \texttt{GeocodingService}.
Each API request requires mandatory parameters including latitude, longitude, and an authentication key, along with optional configuration parameters.
These optional parameters allow customization of the response through \texttt{units} (standard, metric, or imperial measurements),
\texttt{exclude} (response field filtering), and \texttt{lang} (response language selection).
