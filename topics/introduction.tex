\cleardoubleemptypage
\renewcommand*\chapterpagestyle{scrheadings}
\chapter{Introduction}

This project aims to deliver a versatile voice assistant implementation,
offering users the ability to perform various everyday tasks hands-free
and with little effort.

\section{Background and Motivation}
While voice assistants have rapidly gained popularity and their capabilities improve with each new release,
many people still associate them with poor speech-to-text performance and frequent requests for repetition.
This project aims to address these limitations and develop a user-friendly voice assistant solution,
which is production-ready and able to greatly enchance the user's workflow while remaining unobtrusive.
Furthermore, it should expand on the feature set of traditional voice assistans,
offering workspace management on Unix-like systems.

\section{Project Goals}
\begin{itemize}
  \item Efficient processing of queries with a goal of minimal response latency, optimally faster than a second.
  \item Integration into window managers on Unix-like systems, allowing users to control their desktop using voice commands.
  \item Allowing the user to activate audio recording by either pressing a button, or saying a wake word.
  \item Extensive configuration, allowing the user to pick and choose every detail of the system.
\end{itemize}

\section{Naming} \label{sec:naming}
Any modern voice assistant should have a name and a wake word it reacts to;
because of the later implementation of Rust in the project,
"Ferris" was chosen as the name of the voice assistant, referencing
the mascot of the Rust programming language which goes by the same name.
