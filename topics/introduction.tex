\renewcommand*\chapterpagestyle{scrheadings}
\chapter{Introduction}

This project aims to provide a voice assistant implementation that can be used with various frontends,
one such frontend being implemented parallel to this project using the Qt framework \footnote{The Qt Framework \cite{qt}}.

\section{Project Scope}
The project scope encompassed the following objectives:
\begin{itemize}
    \item Efficient processing, with a target response latency of 1 second
    \item Integration into Linux windows managers, allowing users to operate their system using voice commands
    \item Functionality allowing explicit recording and voice activation
    \item Allowing the user to configure which algorithms or APIs they want to use, as well as the output format
\end{itemize}

\section{Background and Motivation}
Voice assistants have experienced steady adoption growth,
though they were often regarded as supplementary rather than essential tools.
This project sought to address this limitation
by developing a voice assistant frontend solution that could enhance users' workflows
while maintaining minimal intrusiveness.
The implementation focused on extending traditional voice assistant capabilities
with advanced features such as workspace management and speech-to-text integration.