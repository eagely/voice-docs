\renewcommand*\chapterpagestyle{scrheadings}


\section{Hardware Overview}
This thesis focuses on the development of an innovative voice assistant based on a Raspberry Pi 4, which combines cutting-edge computing power with high flexibility. The hardware design of the device features a dual power supply, allowing the voice assistant to be operated either through conventional power outlets or rechargeable batteries. This dual energy supply opens up a wide range of applications: the device can be used not only as a stationary unit in a home environment but also offers high mobility and independence from a fixed power source through battery operation. In an era where portable and sustainable technologies are increasingly important, the option to operate the device on battery power is a decisive advantage.


\subsection{User Interface and Hardware Components}
In addition to the powerful Raspberry Pi 4, the hardware of the voice assistant includes an RGB LED that visually indicates the operating status. When the voice assistant is active, the LED lights up in a specified color, providing intuitive feedback to the user. The user interface is further enhanced by two specially designed buttons. The first button, configured as a record button, allows audio to be captured as long as it is held down—an especially useful feature when capturing longer voice commands or dictations, or simply when one does not want the voice to be recorded continuously. The second button, a toggle button, activates the recording function with a single press, ensuring uncomplicated and flexible operation.


\subsection{Benefits of Dual Power Supply}
The decision to also power the voice assistant via a rechargeable battery is forward-thinking for several reasons. On one hand, battery operation enables the device to be used in environments where a fixed power connection is not available, thereby significantly broadening its range of applications. On the other hand, the integration of a battery contributes to energy efficiency and provides a backup power supply in the event of a power outage. Overall, the concept presented in this thesis represents a versatile and robust solution that meets the demands of modern, connected systems in an increasingly mobile world.

\subsection{Display and Visual Feedback}  
In addition to the previously described components, the device is equipped with a display that provides visual feedback on various system statuses. For example, the display can indicate if either of the two buttons is being pressed. This screen is connected to the Raspberry Pi 4 via HDMI, ensuring a reliable and high-quality interface for users to monitor the device’s operations in real time.