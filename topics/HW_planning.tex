\section{Hardware Overview}

The voice assistant is built to run on a Raspberry Pi 4, chosen for its small form factor and flexibility.
While not especially powerful by modern standards, the Raspberry Pi 4 offers enough performance to handle voice input and process
the user's request using remote APIs.
Nevertheless, it is still possible and perfectly acceptable to run the voice assistant on any other Unix-like system,
whether that's a work computer or a more powerful single-board computer made to run machine learning models locally.

The system is designed to work either with a regular power supply or a rechargeable battery;
adding a battery is crucial to keep the system active during power outages.

\subsection{User Interface and Hardware Components}

The printed circuit board includes an RGB LED to signal the current state of the system,
be that lighting up green when recording, or red when an error occurs.
The user can start a recording in three different ways:
\begin{itemize}
    \item Press record button on the touchscreen
    \item Press the dedicated physical recording button
    \item Say the wake word (if enabled)
\end{itemize}
There is also a button --- both on the touchscreen and a physical button ---
that allows the user to toggle wake word detection.

\subsection{Display and Visual Feedback}

A 10.1-inch LCD display should be connected via HDMI to the Raspberry Pi ---
this display allows text output and touchscreen input, especially for configuring settings.
